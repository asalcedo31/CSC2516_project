% Packages to include
\usepackage{helvet}
\usepackage{tikz}
\usepackage{tikz-3dplot}
\usetikzlibrary{backgrounds,3d,shadings,shapes.misc,decorations.pathmorphing,shapes,calc}
%\usepackage{cite}
\usepackage{amssymb,amsmath}
\usepackage{booktabs} % for much better looking tables
\usepackage{mathtools}
%\usepackage{bbding}
\usepackage{graphicx} % support the \includegraphics command and options
\usepackage{subfig}
\usepackage{array} % for better arrays \left(eg matrices\right) in maths
\usepackage{paralist} % very flexible & customisable lists \left(eg. enumerate/itemize, etc.\right)
\usepackage{verbatim} % adds environment for commenting out blocks of text & for better verbatim
\usepackage[labelfont=bf,labelsep=space]{caption} % might need to comment this out in case of conflicts
\usepackage{sectsty} % might need to comment this out in case of conflicts
\usepackage{dblfloatfix} % might need to comment this out in case of conflicts
\usepackage[space]{grffile} % might need to comment this out in case of conflicts
\usepackage{url} % might need to comment this out in case of conflicts
\usepackage[toc]{appendix} % might need to comment this out in case of conflicts
\usepackage[titles,subfigure]{tocloft} % Alter the style of the Table of Contents. Might need to comment this out in case of conflicts
%\usepackage{geometry} % to change the page dimensions
%\geometry{letterpaper}
\usepackage[breakable, theorems, skins]{tcolorbox}
\usepackage{algorithm} % algorithms
\usepackage[noend]{algpseudocode} % algorithms
%\usepackage{hyperref}
\usepackage{pgfplots}
\usepackage{cancel}
\usepackage{bm}
\newcommand{\bs}[1]{\bm{\mathrm{#1}}}

% ---- Definitions ----

% Template for algorithms, source: https://tex.stackexchange.com/questions/163768/write-pseudo-code-in-latex

%\begin{algorithm}
%	\caption{Meta-pruning for weights} \label{alg1}
%	\begin{algorithmic}[1]
%		\Procedure{MyProcedure}{}
%		\State $\textit{stringlen} \gets \text{length of }\textit{string}$
%		\State $i \gets \textit{patlen}$
%		\BState \emph{top}:
%		\If {$i > \textit{stringlen}$} \Return false
%		\EndIf
%		\State $j \gets \textit{patlen}$
%		\BState \emph{loop}:
%		\If {$\textit{string}(i) = \textit{path}(j)$}
%		\State $j \gets j-1$.
%		\State $i \gets i-1$.
%		\State \textbf{goto} \emph{loop}.
%		\State \textbf{close};
%		\EndIf
%		\State $i \gets i+\max(\textit{delta}_1(\textit{string}(i)),\textit{delta}_2(j))$.
%		\State \textbf{goto} \emph{top}.
%		\EndProcedure
%	\end{algorithmic}
%\end{algorithm}



% Coloured boxes for highlighting text, e.g. for summaries, theorems, etc.
\DeclareRobustCommand{\colourbox}[2][gray!20]{%
	\begin{tcolorbox}[   %% Adjust the following parameters at will.
		breakable,
		left=0pt,
		right=0pt,
		top=0pt,
		bottom=0pt,
		colback=#1,
		colframe=#1,
		width=\dimexpr\textwidth\relax, 
		enlarge left by=0mm,
		boxsep=5pt,
		arc=0pt,outer arc=0pt,
		]
		#2
	\end{tcolorbox}
}


% ---- My definitions ----
%\DeclareMathOperator{\pv}{pv}
%\DeclareMathOperator{\sign}{sign}
%\DeclareMathOperator{\ber}{ber}
%\DeclareMathOperator{\bei}{bei}
%\newcommand{\vect}[1]{\mathbf{#1}}
%\newcommand{\nhat}{\mathbf{\hat{n}}}
\newcommand{\vect}[1]{\vec{#1}}
\newcommand{\nhat}{{\hat{n}}}
\newcommand{\nvec}{{\hat{n}}}
\newcommand{\ver}[1]{\hat{#1}}
%\newcommand{\matr}[1]{\mathbf{#1}}
\newcommand{\matr}[1]{\bs{#1}}
\newcommand{\abs}[1]{\left \lvert #1 \right\rvert }
\newcommand{\norm}[1]{ \left \lVert #1 \right \rVert} 
%\newcommand{\pvint}{\pv\!\!\int}
\newcommand{\pvint}{\dashint}
\newcommand{\intR}{\int_{-\infty}^{+\infty}}
\newcommand{\pvintR}{\pvint_{-\infty}^{+\infty}}
\newcommand{\laplace}[1]{\mathscr{L}\{#1\}}
\newcommand{\ilaplace}[1]{\mathscr{L}^{-1}\{#1\}}
\newcommand{\blaplace}[1]{\mathscr{L}_b\{#1\}}
\newcommand{\fourier}[1]{\mathscr{F}\{#1\}}
\newcommand{\ifourier}[1]{\mathscr{F}^{-1}\{#1\}}
\renewcommand{\Re}[1]{\mathbb{R}\mathrm{e}\left \{#1\right\} }
\renewcommand{\Im}[1]{\mathbb{I}\mathrm{m}\left \{#1\right\} }
\newcommand{\Real}{\mathbb{R}}
\newcommand{\Complex}{\mathbb{C}}
% Sets
\newcommand{\complem}[1]{#1^{\rm C}}
\newcommand{\expon}[1]{\operatorname{e}^{\,#1}}
\newcommand{\ns}{\!\!\!\!}
\newcommand{\pref}[1]{(\ref{#1})}
\newcommand{\junk}[1] {}
\newcommand{\missing}[1]{\vspace*{\parskip}
			\centerline{\framebox{\begin{minipage}{0.9\columnwidth}{\large\bf\noindent #1}
				\end{minipage}}}
			\vspace*{\parskip}}


% Principal value integral sign
\def\Xint#1{\mathchoice
{\XXint\displaystyle\textstyle{#1}}%
{\XXint\textstyle\scriptstyle{#1}}%
{\XXint\scriptstyle\scriptscriptstyle{#1}}%
{\XXint\scriptscriptstyle\scriptscriptstyle{#1}}%
\!\int}
\def\XXint#1#2#3{{\setbox0=\hbox{$#1{#2#3}{\int}$}
\vcenter{\hbox{$#2#3$}}\kern-.5\wd0}}
\def\ddashint{\Xint=}
\def\dashint{\Xint-}

\newcommand*\widebar[1]{%
  \hbox{%
    \vbox{%
      \hrule height 0.5pt % The actual bar
      \kern0.3ex%         % Distance between bar and symbol
      \hbox{%
        \kern-0.05em%      % Shortening on the left side
        \ensuremath{#1}%
        \kern-0.05em%      % Shortening on the right side
      }%
    }%
  }%
} 


\renewcommand{\epsilon}{\varepsilon}

\newcommand{\dbd}[2]{\ensuremath{\dfrac{d#1}{d#2}}}
\newcommand{\ddt}[1]{\ensuremath{\dfrac{d#1}{dt}}}
\newcommand{\ddx}[1]{\ensuremath{\dfrac{d#1}{dx}}}
\newcommand{\ddy}[1]{\ensuremath{\dfrac{d#1}{dy}}}
\newcommand{\ddz}[1]{\ensuremath{\dfrac{d#1}{dz}}}

\newcommand{\pardt}[1]{\ensuremath{\dfrac{\partial#1}{\partial t}}}
\newcommand{\pardx}[1]{\ensuremath{\dfrac{\partial#1}{\partial x}}}
\newcommand{\pardy}[1]{\ensuremath{\dfrac{\partial#1}{\partial y}}}
\newcommand{\pardz}[1]{\ensuremath{\dfrac{\partial#1}{\partial z}}}

\newcommand{\pardtsq}[1]{\ensuremath{\dfrac{\partial^2#1}{\partial t^2}}}
\newcommand{\pardxsq}[1]{\ensuremath{\dfrac{\partial^2#1}{\partial x^2}}}
\newcommand{\pardysq}[1]{\ensuremath{\dfrac{\partial^2#1}{\partial y^2}}}
\newcommand{\pardzsq}[1]{\ensuremath{\dfrac{\partial^2#1}{\partial z^2}}}

\newcommand{\curl}[1]{\ensuremath{\nabla\times\bs{#1}}}
\newcommand{\tensor}[1]{\ensuremath{\bar{\bar{#1}}}}

\newcommand{\figref}[1]{Figure~\ref{#1}}
\newcommand{\algoref}[1]{Algorithm~\ref{#1}}
\newcommand{\secref}[1]{Section~\ref{#1}}
\newcommand{\tabref}[1]{Table~\ref{#1}}

\newcommand{\opL}{\mathcal{L}} % L operator
\newcommand{\opK}{\mathcal{K}} % K operator

% Theorems, laws, definitions environments
\newtheorem{defn}{Definition}
\newtheorem{thm}{Theorem}
\newtheorem{exm}{Example}
\newtheorem{rmk}{Remark}


\usepackage{listings}
\usepackage{color}

\definecolor{mygreen}{RGB}{28,172,0} % color values Red, Green, Blue
\definecolor{mylilas}{RGB}{170,55,241}

\lstset{language=Matlab,%
	%basicstyle=\color{red},
	basicstyle=\ttfamily,
	breaklines=true,%
	morekeywords={matlab2tikz},
	%keywordstyle=\color{blue},%
	morekeywords=[2]{1}, keywordstyle=[2]{\color{black}},
	identifierstyle=\color{black},%
	stringstyle=\color{mylilas},
	commentstyle=\color{mygreen},%
	showstringspaces=false,%without this there will be a symbol in the places where there is a space
	numbers=left,%
	numberstyle={\tiny \color{black}},% size of the numbers
	numbersep=9pt, % this defines how far the numbers are from the text
	emph=[1]{for,end,break},emphstyle=[1]\color{blue}, %some words to emphasise
	%emph=[2]{word1,word2}, emphstyle=[2]{style},    
}

